\section{Executive summary}
You should read this section if you want to get a flavour of the SPI product set.

\subsection{Standard use case}

Your group develops and uses a number of libraries written in C or C++ which will then be used in application development.
In general C++ is not the language of choice for application development.
In addition you wish to separate the development of the calculation libraries (hereafter named as analytics libraries) from the application development.

The analytics libraries exist - either developed in-house or externally.
The idea is that you wrap these libraries with an extra layer in order to satisfy a variety of different user interface requirements.

SPI provides interface-neutral text-based configuration file format that will describe the low-level libraries you are trying to wrap, and from this information generate the code required by the different interface technologies.

The interface code is generated automatically from the configuration files.
Not only do we generate code, but we also generate a user guide for the product which can be produced before any code is fully compiled.
Thus you can provide a specification for discussion with your clients at an early stage in any project.

\subsection{Features provided by use of SPI}

Using SPI you can provide Excel, Python and .NET wrappers of your analytics library.
The interface via Excel is recognisably the same as the interface via Python or .NET.
There is also a special C++ wrapper of your analytics library if you need to release a C++ version of your library.

The C++ wrapper supports the concepts of object orientation (via classes and inheritance), serialization, logging and replay.
The same features are then reflected in the Excel, Python and .NET wrappers.
The Excel and Python wrappers also support the concept of array functions by which we mean that if you pass in an array of inputs to a function which expects a scalar and returns a scalar, then the function will accept the array and return an array of outputs one for each input provided.

Serialization means that classes are can be converted to and from text formats.
Logging and replay means that you can record a log of all function calls to the library and subsequently replay that log of function calls independently of the application that created the log file.

\subsection{Future developments}

This a list of possible future enhancements to the SPI product set.
\begin{itemize}
	\item Binary serialization.
	\item Remote function calls.
	\item Support for other languages, e.g. Java, Perl etc.
\end{itemize}

\subsection{Current examples of use}

The SPI package comes with some sample code which demonstrates the product in action.
These are real analytics (using open source code) but the goal is not to provide an analytics library that you could use in production.
Rather the goal is to show the language features of SPI in action.
Thus the sample code will provide some very basic credit and rates analytics functionality - build a discount curve, compute market rates and fair value for some simple products.

In addition I have developed some software for the analysis of bridge hands (bridge is a card game).
The low-level analysis software is an open source library written in C++.
Using SPI this software is available via Excel and Python which you can use to build applications which use the low-level analysis tools.
